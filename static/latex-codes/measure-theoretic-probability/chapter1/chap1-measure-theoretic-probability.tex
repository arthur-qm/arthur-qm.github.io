\documentclass{article}
\usepackage{amsmath,amssymb,amsthm}
\usepackage{etoolbox}
\usepackage{tikz}
\usepackage{enumitem}

%-------------------------------
% Theorem Environments Setup
%-------------------------------
\theoremstyle{definition}
\newtheorem{theorem}{Theorem}[section]
\newtheorem{lemma}[theorem]{Lemma}
\newtheorem{proposition}[theorem]{Proposition}
\newtheorem{corollary}[theorem]{Corollary}
\newtheorem{definition}[theorem]{Definition}
\newtheorem{remark}[theorem]{Remark}
\newtheorem{example}[theorem]{Example}
\newtheorem{exercise}[theorem]{Exercise}



%-------------------------------
% Hyperlink Reference Command
%-------------------------------
\newcommand{\tref}[1]{\ref{#1}}

%-------------------------------
% Document Content
%-------------------------------
\begin{document}

\section{Introduction}

For me, the best way to learn something is to do it by explaining it to other people. With this in mind,
I decided to make a website which supports blog posts so that I can document what I'm learning about.
So you'll be pretty much learning together with me. One of the things I was really excited about learning was
measure theoretic probability.

If you've already had regular probability or statistics courses, you might be questioning if it is
reasonable to study it all over again from the viewpoint of measure theory.

And you might be right. Maybe you really won't need it. In fact, many statisticians don't even use it.
Depending on what you need to apply your probability or statistics knowledge to, there's not a lot of
things you'll be missing out on if you don't learn it.

So why does it even exist? Well, it's just the natural way of formalizing probability theory. If you
actually want to prove statements involving probability, then learning it from a measure theoretic
viewpoint is the way to go.

For example, you might say that an event is ``a collection of possible outcomes of a random experiment''.
But this is too sloppy. There's no actual mathematical framework behind it.

Another clear example are the convergence theorems. Measure theory helps defining lots of modes of
convergence so that we know precisely what is being told when you say e.g.\ some random variable
converges to another random variable.

TODO: show specific example of convergence

I'll be using Rick Durrett's book ``Probability: Theory and Examples'' as a guide, and I'll be following it closely.
I'll also try to introduce some of my own ideas/some stuff I find online.

So without further ado, let's begin.

\section{Probability Spaces}

\begin{definition}[\textbf{Probability Space}]
    Let $\Omega$ be a non-empty set, $\mathcal{F}$ a $\sigma$-algebra
    on $\Omega$ and $P$ a measure on the measurable space $(\Omega, \mathcal{F})$ so that $P(\Omega)=1$. We
    call the measure space $(\Omega, \mathcal{F}, P)$ a probability space.
\end{definition}

It follows that the domain of $P$ is $\mathcal{F}$. So we are assigning each element of $\mathcal{F}$ a value
ranging from $0$ to $1$. In the "naive" probability theory, we assign probabilities to events, so that in our
more refined theory, the events are precisely the elements of $\mathcal{F}$.

For this reason, $\mathcal{F}$ is called the \textit{event space}. The set $\Omega$ is called the \textit{sample space}
and $P$ is called the \textit{probability function}.

We can say that the definition of a probability space is a mathematical construct that provides a formal model
of a \textit{random process} or \textit{experiment}.

When an experiment is conducted, it results in exactly one outcome $\omega \in \Omega$. All the events $S$ in the event space
$\mathcal{F}$ that contain the selected outcome $\omega$ are said to "have occurred". This means that when we model our
probability space, the probability function $P$ must be defined in a way that if the experiment were repeated arbitrarily
many times, the number of occurrences of each event as a fraction of the total number of experiments, will (most likely)
tend towards the probability assigned to that event.

\begin{example}[\textbf{Throw of a standard die}]
    $\Omega = \{1, 2, 3, 4, 5, 6\}$, $\mathcal{F} = \mathcal{P}(\Omega)$
    (i.e.\ the set of all subsets of $\Omega$) and $P(\{ i \}) = 1/6$ for all $i \in \Omega$. Notice that
    this indeed defines a probability space (i.e.\ $\mathcal{F}$ is indeed a $\sigma$-algebra on $\Omega$ and although
    I didn't specify the value of $P$ at all of $\mathcal{F}$, it is clear that there exists a unique probability measure
    on $(\Omega, \mathcal{F})$ which satisfies what I specified, namely $P(S) = |S|/6$, where $|S|$ is the number of elements
    of some $S \in \mathcal{F}$).
\end{example}

\begin{example}[\textbf{Toss of a fair coin}]
    Take $\Omega = \{ \text{H}, \text{T} \}$, $\mathcal{F} = \mathcal{P}(\Omega)$ and $P$
    so that $P(\emptyset) = 0$, $P(\{ \text{H} \}) = 1/2$, $P(\{ \text{T} \}) = 1/2$, and $P(\{ \text{H}, \text{T} \}) = 1$.
\end{example}

\begin{example}[\textbf{Uniform distribution}]
    Let $\Omega = [0, 1]$, $\mathcal{F} = \mathcal{B}_{\Omega}$ (i.e.\ the $\sigma$-algebra
    of Borel sets on $\Omega$) and $P$ be the Borel measure on $[0, 1]$. In this case, for each $a, b \in \Omega$ with $a \leq b$,
    we have $P((a, b)) = b - a$.
\end{example}

\section{Random elements and distributions}

\begin{definition}[\textbf{Random element}]
    A random element over a probability space $(\Omega, \mathcal{F}, P)$ is a measurable function $X : (\Omega, \mathcal{F}) \to (S, \mathcal{S})$ where $(S, \mathcal{S})$ is an arbitrary measurable space. We say that the probability that $X$ takes some value in $T \in \mathcal{S}$ is $P(X \in T) = P(\{\omega \in \Omega : X(\omega) \in T \}) = P(X^{-1}(T))$ where
    the $X \in T$ is just syntactic sugar for $X^{-1}(T)$. In most cases, we'll take $(S, \mathcal{S})$ to be $(\mathbb{R}, \mathcal{B}_{\mathbb{R}})$, in which case we'll call $X$ a random variable (or r.v.). If $(S, \mathcal{S}) = (\mathbb{R}^n, \mathcal{B}_{\mathbb{R}^n})$, we say $X$ is a random vector.
\end{definition}

\begin{example}[\textbf{Another throw of a die}]
    Consider our previous throw of a standard die example. Let $X$ be the function which maps each $\omega \in \Omega$ to $\omega$. Clearly, $X$ is a measurable map from $(\Omega, \mathcal{F})$ to $(\mathbb{R}, \mathcal{B}_{\mathbb{R}})$ so that it is a random variable. Informally, this means that when we run our experiment, there is $1/6$ chance of $X$
    being equal to $i$ for each $i \in \Omega$. Indeed, $P(X=1) = P(X^{-1}(\{1\})) = P(\{1 \}) = 1/6$.
\end{example}

Notice the use of the notation $P(X=1)$. If you want to be pedantic, you are right to say that we haven't defined it properly.
But don't be like that. What I meant was clearly $P(X \in \{1\})$ (which I've already defined). More precisely, whenever I have $P$ of some expression involving random elements satisfying something on them, I mean the set of all $\omega \in \Omega$ for which the
expression involving the random element is true. Of course, we must verify that such a set is measurable, but that's generally clear from the context.

\begin{example}[\textbf{Indicator function}]
    Let $A \in \mathcal{F}$. We define the \emph{indicator function} $\mathbf{1}_{A}$ of $A$ from $\Omega$ to $\mathbb{R}$ as follows. If $\omega \in A$, then $\mathbf{1}_{A}(\omega) = 1$. If this isn't the case, $\mathbf{1}_{A}(\omega) = 0$. It is easy to check that $\mathbf{1}_{A}$ is a r.v.
\end{example}

By the way, I'll not bother being overly descriptive. For example, if I just say that $X$ is a random element, you can immediately assume that what I mean is $X : (\Omega, \mathcal{F}) \to (S, \mathcal{S})$ (because that's the way I defined it).


\begin{definition}[\textbf{Distribution induced by a random element}]
    Let $X$ be a random element. Then, the function $\mu : \mathcal{S} \to \mathbb{R}^+$ defined by $\mu(A) = P(X \in A)$ is the distribution induced by $X$. Notice how $\mu$ is a measure over $(S, \mathcal{S})$.
\end{definition}

\begin{definition}[\textbf{Distribution function induced by a random variable}]
    Let $X$ be a random variable. Then, we define its induced distribution function $F : \mathbb{R} \to [0, 1]$ by $F(x) = P(X \leq x)$.
\end{definition}

Notice how we define both "distribution" and "distribution function" induced by a r.v. so be careful.


\begin{theorem}[\textbf{Properties of the distribution function}]
    Any distribution function $F$ has the following properties:
    \begin{itemize}
        \item $F$ is nondecreasing
        \item $\lim\limits_{x \to \infty} F(x) = 1$, $\lim\limits_{x \to -\infty} F(x) = 0$
        \item $F$ is right continuous, i.e.\ $\lim\limits_{y \downarrow x} F(y) = F(x)$
        \item If $F^{-}(x) = \lim\limits_{y \uparrow x} F(y)$, then $F^{-}(x) = P(X < x)$
        \item $P(X=x) = F(x) - F^{-}(x)$
    \end{itemize}
\end{theorem}


\myproof{
    I believe in you, you can do it!
}

\begin{theorem}[\textbf{Existence of r.v. given distribution function}]\label{thm:existence-rv}
    Let $F : \mathbb{R} \to \mathbb{R}$ be a function satisfying all properties listed in the previous theorem. Then, there exists a probability space $(\Omega, \mathcal{F}, P)$ and a r.v.\ $X$ over it such that the distribution function of $X$ is precisely $F$.
\end{theorem}

\myproof{
    (Sketch) Just take $\Omega = (0, 1)$, $\mathcal{F} = $ borel sets contained in $\Omega$, $P$ the borel measure and
    $X(\omega) = \sup \{ y \in \mathbb{R} : F(y) < \omega\}$.
}

\begin{remark}
    It may happen that $F$ is not invertible. Even so, we shall write $F^{-1}(x)$ meaning $\sup \{ y \in \mathbb{R} : F(y) < x \}$ so that our $X(\omega)$ on the proof above is $F^{-1}(\omega)$.
\end{remark}

\begin{remark}
    Let $X, Y$ be r.v.'s (when I mention multiple random elements, it's implicit that they're defined on the same probability space and have the same codomain). If the distribution of $X$ is the same as the distribution of $Y$ (which happens iff they have the same distribution function, due to the Lebesgue-Stieltjes theorem), we say $X$ and $Y$ are equal in distribution and write $X \stackrel{d}{=} Y$.
\end{remark}

\begin{definition}[\textbf{Density function}]
    When the distribution function $F$ can be written as
    \[ F(x) = \int\limits_{-\infty}^x f(t) dt \]
    we say that $X$ has density function $f$. In fact, we can start with an integrable function $f$ satisfying $f \geq 0$ and $\int_{-\infty}^{\infty} f(t) dt = 1$.
\end{definition}

\begin{definition}[\textbf{Discrete measure}]
    A probability measure $P$ is said to be discrete if there is a countable set $S$ with $P(S^c) = 0$.
\end{definition}

\begin{proposition}
    The set of discontinuities of a distribution function $F$ is countable.
\end{proposition}

\begin{example}[\textbf{Dense discontinuities}]
    Although the set of discontinuities of $F$ is countable, it may be dense. Indeed, let $\{ q_n \}_{n=1}^\infty$ be an enumeration of the rationals and let $\{ \alpha_n \}_{n=1}^\infty$ be a sequence of positive real numbers such that $\sum_{n=1}^\infty \alpha_n = 1$. Define
    \[ F(x) = \sum\limits_{n=1}^\infty \alpha_n \cdot \mathbf{1}_{[q_n, \infty)}(x) \]
    then $F$ is clearly a distribution function of some random variable $X$.
\end{example}

\begin{exercise}
    Let $(\Omega, \mathcal{F}, P)$ be a probability space and $X$ be a r.v.\ over it with distribution function $F$. Let $Y = F \circ X$. Clearly $Y$ is a r.v.\ Show that $P(Y \leq y) = y$ for each $y \in [0,1]$.
\end{exercise}

\end{document}