\documentclass{article}
\usepackage{amsmath,amssymb,amsthm}
\usepackage{etoolbox}
\usepackage{tex4ht}
\usepackage{enumitem}

%-------------------------------
% Theorem Environments Setup
%-------------------------------


% Define theorem-like environments
\theoremstyle{definition}
\newtheorem{theorem}{Theorem}[section]
\newtheorem{lemma}[theorem]{Lemma}
\newtheorem{proposition}[theorem]{Proposition}
\newtheorem{corollary}[theorem]{Corollary}
\newtheorem{definition}[theorem]{Definition}
\newtheorem{remark}[theorem]{Remark}
\newtheorem{example}[theorem]{Example}
\newtheorem{exercise}[theorem]{Exercise}

%-------------------------------
% Custom Proof Environment (unchanged)
%-------------------------------
\DeclareRobustCommand{\myproof}[1]{%
  \HCode{<details style="margin:1.5rem 0;border:2px solid \#3b82f6;border-radius:8px;background:white;">}%
  \HCode{<summary style="cursor:pointer;padding:1rem;background:\#e0f2fe;color:\#1d4ed8;font-weight:bold;">Proof</summary>}%
  \HCode{<div style="padding:1rem;">}%
  #1%
  \HCode{</div></details>}%
}

%-------------------------------
% Hyperlink Reference Command (unchanged)
%-------------------------------
\newcommand{\tref}[1]{\ref{#1}}


%-------------------------------
% Document Content
%-------------------------------
\begin{document}

\section{Number Theory Basics}

\begin{theorem}[Fundamental Theorem of Arithmetic]\label{thm:fundamental}
  Every integer greater than 1 either is a prime itself or can be represented as the product of primes, unique up to order.
\end{theorem}

\myproof{
  test
}

\begin{lemma}[Euclid's Lemma]\label{lem:euclid}
  If a prime \( p \) divides the product \( ab \), then \( p \) divides \( a \) or \( p \) divides \( b \).
\end{lemma}

\begin{corollary}\label{cor:infinite-primes}
  There are infinitely many primes.
\end{corollary}


Reference examples: \ref{thm:fundamental}, \ref{lem:euclid}, and \ref{cor:infinite-primes}

\end{document}
